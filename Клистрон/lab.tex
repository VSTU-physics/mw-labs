\documentclass[a4paper]{article}
\usepackage[russian]{babel}
\usepackage[utf8]{inputenc}
\usepackage[margin=1cm,left=2cm]{geometry} % сделать по ГОСТУ
\usepackage{tikz}
\begin{document}
	% не забыть вставить титульник
	\section*{Введение}
	В этой работе определяются характеристики отражательного клистрона К-97Р. Определяются его рабочая частота и добротность его резонатора.
	
	\section{Теоретическая часть}
	\subsection{Устройство отражательного клистрона}
	Катод, резонатор, отражатель и мыкающиеся электроны
	\subsection{Принцип работы отражательного клистрона}
	Модуляция, группировка, возвращение в нужную фазу. Кинетическая теория клистрона, условие генерации.
	
	\section{Методика проведения измерений}
	\subsection{Описание установки}
	Так как мы поссорились с нужным клистроном с волноводным переходником и он больше не хочет работать дольше пяти минут, то мы изменили ему с К-54. Все дальнейшие измерения проводились на последнем.

	Для определения характеристик клистрона необходимо узнать при каких напряжениях на резонаторе и отражателе начинается генерация. Для этого необходимо подключить клистрон к измерителю мощности.

	По-моему 2 схемы, учитывая многочисленность компонентов, рисовать бессмысленно и можно просто остановиться на функциональной.

	Далее в пределах одной зоны можно найти основную частоту и засчёт электронной перестройки найти добротность.
	
	\subsection{Описание приборов}
	\subsubsection{Блок питания клистрона}
	\begin{tikzpicture}[
  scale=.1,
  controlpanels/.style={yellow!30!brown!20!,rounded corners,draw=black,thick},
  screen/.style={green!50!black!60!,draw=black,thick},
  trace/.style={green!60!yellow!40!, ultra thick},
  smallbutton/.style={white,draw=black, thick},
  axes/.style={thick}]
  	% контур
	\draw[rounded corners,draw=black,thick](0,0)
    rectangle (100,100);
    % клеммы
    \draw[draw=black,thick](30,90) circle (4);
    \draw[draw=black,thick](30,90) circle (2);
    \draw[draw=black,thick](40,90) circle (4);
    \draw[draw=black,thick](40,90) circle (2);
    \draw[draw=black,thick](50,90) circle (4);
    \draw[draw=black,thick](50,90) circle (2);
    \draw[draw=black,thick](60,90) circle (4);
    \draw[draw=black,thick](60,90) circle (2);
    \draw[draw=black,thick](70,90) circle (4);
    \draw[draw=black,thick](70,90) circle (2);
    \draw[draw=black,thick](80,80) circle (3);
    \draw[draw=black,thick](80,80) circle (1.5);
    \draw[draw=black,thick](90,80) circle (3);
    \draw[draw=black,thick](90,80) circle (1.5);
    % тумблеры
    % переключение вольтметра
    \draw[black,thick](50, 70) circle (3);
    \draw[black,thick](50, 70) circle (2);
    \fill[fill=white,rounded corners,draw=black,thick](49,69)
    rectangle (56,71);
    % включение напряжений + индикаторы
    \draw[black,thick](80, 50) circle (3);
    \draw[black,thick](90, 50) circle (3);
    \draw[black,thick](90, 50) circle (2);
    \fill[fill=white,rounded corners,draw=black,thick](89,51)
    rectangle (91,44);

    \draw[black,thick](80, 35) circle (3);
    \draw[black,thick](90, 35) circle (3);
    \draw[black,thick](90, 35) circle (2);
    \fill[fill=white,rounded corners,draw=black,thick](89,36)
    rectangle (91,29);

    \draw[black,thick](80, 20) circle (3);
    \draw[black,thick](90, 20) circle (3);
    \draw[black,thick](90, 20) circle (2);
    \fill[fill=white,rounded corners,draw=black,thick](89,21)
    rectangle (91,14);
    % основной тумблер
    \draw[black,thick](20, 10) rectangle (25,20);
    \draw[black](20, 15) -- (25,15);

    % вольтметр
    \draw[rounded corners,draw=black,thick](30,10)
    rectangle (70,60);
	\end{tikzpicture}
	\subsubsection{Измеритель мощности}
	\subsubsection{Анализатор спектра}

	\subsection{Проведение измерений}
	\subsubsection{Определение зон генерации}

	\subsubsection{Определение рабочей частоты и добротности}
\end{document}