\documentclass[a4paper]{article}
\usepackage[russian]{babel}
\usepackage[utf8]{inputenc}
\usepackage[margin=1cm,left=2cm]{geometry} % сделать по ГОСТУ
\usepackage{tikz}
\usetikzlibrary{circuits.ee.IEC}
\begin{document}
	% не забыть вставить титульник
	\section*{Введение}
	В этой работе определяются характеристики отражательного клистрона К-97Р. Определяются его рабочая частота и добротность его резонатора.
	
	\section{Теоретическая часть}
	\subsection{Устройство отражательного клистрона}
	Катод, резонатор, отражатель и мыкающиеся электроны
	\subsection{Принцип работы отражательного клистрона}
	Модуляция, группировка, возвращение в нужную фазу. Кинетическая теория клистрона, условие генерации.
	
	\section{Методика проведения измерений}
	\subsection{Описание установки}
	Так как мы поссорились с нужным клистроном с волноводным переходником и он больше не хочет работать дольше пяти минут, то мы изменили ему с К-54. Все дальнейшие измерения проводились на последнем.

	Для определения характеристик клистрона необходимо узнать при каких напряжениях на резонаторе и отражателе начинается генерация. Для этого необходимо подключить клистрон к измерителю мощности.

	По-моему 2 схемы, учитывая многочисленность компонентов, рисовать бессмысленно и можно просто остановиться на функциональной.

	Далее в пределах одной зоны можно найти основную частоту и засчёт электронной перестройки найти добротность.
	
	\subsection{Описание приборов}
	\subsubsection{Блок питания клистрона}
    \begin{figure}[h]
    \center
	\begin{tikzpicture}[
  scale=.08 ,
  circuit ee IEC]
  	% контур
	\draw[rounded corners,draw=black,thick](0,0)
    rectangle (100,100);
    % клеммы
    \draw[draw=black,thick](30,90) circle (4);
    \draw[draw=black,thick](30,90) circle (2);
    \draw[draw=black,thick](40,90) circle (4);
    \draw[draw=black,thick](40,90) circle (2);
    \draw[draw=black,thick](50,90) circle (4);
    \draw[draw=black,thick](50,90) circle (2);
    \draw[draw=black,thick](60,90) circle (4);
    \draw[draw=black,thick](60,90) circle (2);
    \draw[draw=black,thick](70,90) circle (4);
    \draw[draw=black,thick](70,90) circle (2);
    \draw[draw=black,thick](80,80) circle (3);
    \draw[draw=black,thick](80,80) circle (1.5);
    \draw[draw=black,thick](90,80) circle (3);
    \draw[draw=black,thick](90,80) circle (1.5);
    % тумблеры
    % переключение вольтметра
    \draw[black,thick](50, 78) circle (3);
    \draw[black,thick](50, 78) circle (2);
    \fill[fill=white,rounded corners,draw=black,thick](49,77)
    rectangle (56,79);
    % включение напряжений + индикаторы
    \draw[black,thick](80, 50) circle (3);
    \draw[black,thick](90, 50) circle (3);
    \draw[black,thick](90, 50) circle (2);
    \fill[fill=white,rounded corners,draw=black,thick](89,51)
    rectangle (91,44);

    \draw[black,thick](80, 35) circle (3);
    \draw[black,thick](90, 35) circle (3);
    \draw[black,thick](90, 35) circle (2);
    \fill[fill=white,rounded corners,draw=black,thick](89,36)
    rectangle (91,29);

    \draw[black,thick](80, 20) circle (3);
    \draw[black,thick](90, 20) circle (3);
    \draw[black,thick](90, 20) circle (2);
    \fill[fill=white,rounded corners,draw=black,thick](89,21)
    rectangle (91,14);
    % основной тумблер
    \draw[black,thick](20, 10) rectangle (25,20);
    \draw[black](20, 15) -- (25,15);
    \node[scale=1] at (22.5,12.5) {$0$};
    \node[scale=1] at (22.5,17.5) {$1$};

    % потенциометр отражателя
    \draw[draw=black,thick](20,80) circle (5);

    % потенциометр резонатора
    \fill[fill=white,rounded corners,draw=black,thick](100,79)
    rectangle (106,81);


    % вольтметр
    \draw[rounded corners,draw=black,thick](30,10) rectangle (70,60);
    \draw[draw=black,thick](32,22) rectangle (68,58);
    \begin{scope}[xshift=50cm,yshift=16cm,scale=25]
        \foreach \i in {60,63,...,120} \draw (\i:1.15)--(\i:1.20);
        \foreach \i in {60,75,...,120} \draw[thick] (\i:1.15)--(\i:1.25);
        \node[scale=.7] at (60:1.05) {$400$};
        \node[scale=.7] at (75:1.35) {$300$};
        \node[scale=.7] at (90:1.35) {$200$};
        \node[scale=.7] at (105:1.35) {$100$};
        \node[scale=.7] at (120:1.05) {$0$};
        \draw[very thick] (95:0.25) -- (95:1);
        \draw (95:1) -- (95:1.20);
        \node[scale=1.5] at (90:0.7) {$\mathrm{V}$};
    \end{scope}

    % подписи
    % клеммы
    \node[scale=1.25] at (30,83) {$-$};
    \node[scale=1.25] at (40,83) {$+$};
    \node[scale=1.25] at (35,78) {$190$};

    \draw (50,84.5) to (50, 83) node[ground,rotate=-90,xshift=3.9ex,scale=6] {};

    \node[scale=1.25] at (60,83) {$-$};
    \node[scale=1.25] at (70,83) {$+$};
    \node[scale=1.25] at (65,78) {$350$};
    \node[scale=1] at (84,75) {$\sim 6.3$};

    % обозначения
    \draw (37, 93) -- (20,105) -- (15,105) node[above right] {$1$};
    \draw (30, 94) -- (30,98);

    \draw (47, 93) -- (30,105) -- (25,105) node[above right] {$2$};

    \draw (67, 93) -- (50,105) -- (45,105) node[above right] {$3$};
    \draw (60, 94) -- (60,98);

    \draw (16,82) -- (-5, 90) -- (-10, 90) node[above right] {$4$};
    
    \draw (47,77) -- (-5, 60) -- (-10, 60) node[above right] {$5$};

    \draw (82, 82) -- (105,105) -- (110,105) node[above left] {$6$};
    \draw (90, 83) -- (90,90);

    \draw (106, 80) -- (110, 90) -- (115,90) node[above left] {$7$};

    \draw (93, 50) -- (105, 40) -- (110,40) node[above left] {$8$};
    \draw (93, 35) -- (105, 40);
    \draw (93, 20) -- (105, 40);

    \draw (22.5, 10) -- (15, -10) -- (10,-10) node[above right] {$9$};

    \draw (50, 10) -- (70, -10) -- (75,-10) node[above left] {$10$};
	\end{tikzpicture}
    \end{figure}
	\subsubsection{Измеритель мощности}
	\subsubsection{Анализатор спектра}

	\subsection{Проведение измерений}
	\subsubsection{Определение зон генерации}

	\subsubsection{Определение рабочей частоты и добротности}
\end{document}