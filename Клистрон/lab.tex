\documentclass[a4paper,14pt]{extarticle}
\usepackage[russian]{babel}
\usepackage[colorlinks=False]{hyperref}
\usepackage[utf8]{inputenc}
\usepackage[margin=1.5cm,left=2cm,right=1cm]{geometry} % сделать по ГОСТУ
\usepackage{indentfirst}
\usepackage{graphicx}
\usepackage{amsmath}
\usepackage{multirow}
\usepackage{pgfplots}
\usepgfplotslibrary{fillbetween}

\usepackage{tikz}
\usetikzlibrary{circuits.ee.IEC}

\usepackage{circuitikz}
\ctikzset{bipoles/length=.8cm}

\tikzset{component/.style={draw,thick,circle,fill=white,minimum size =0.8cm,inner sep=0pt}}

\renewcommand{\small}{\fontsize{12pt}{14.4pt}\selectfont}
\usepackage{caption}
\DeclareCaptionLabelFormat{figure}{Рисунок #2}
\DeclareCaptionLabelFormat{table}{Таблица #2}
\DeclareCaptionLabelSeparator{sep}{~---~}
\captionsetup{labelsep=sep,justification=centering,font=small}
\captionsetup[figure]{labelformat=figure}
\captionsetup[table]{labelformat=table}

\usepackage{titlesec}
\titleformat{\section}
    {\centering\normalsize}
    {\thesection}
    {1em}{}
\titleformat{\subsection}
    {\centering\normalsize}
    {\thesubsection}
    {1em}{}
\titleformat{\subsubsection}
    {\centering\normalsize}
    {\thesubsubsection}
    {1em}{}

\titlespacing*{\section}{\parindent}{*4}{*1}
\titlespacing*{\subsection}{\parindent}{*4}{*1}
\titlespacing*{\subsubsection}{\parindent}{*4}{*1}

\renewcommand{\tan}{\mathrm{tg\,}}
\renewcommand{\baselinestretch}{1.5}
\begin{document}
	\begin{titlepage}
        \begin{center}
          Министерство образования и науки Российской Федерации \\
          Федеральное государственное бюджетное образовательное \\
          учреждение высшего профессионального образования \\
          <<Волгоградский государственный технический университет>> \\
          Факультет электроники и вычислительной техники\\
          Кафедра <<Физика>>
        \end{center}
        \vspace{9em}
        \begin{center}
          \large
            Методические указания к лабораторной работе\\
            <<Исследование характеристик клистрона К-97>>
        \end{center}
        \vspace{5em}
        \begin{flushright}
          \begin{minipage}{.40\textwidth}
            Выполнили:\\
            студенты группы Ф-1н\\
            Абдрахманов В. Л.\\
            Аликов С. А.\\
            \vspace{1em}\\
            Проверил:\\
            д.ф.-м.н., профессор\\
            Шеин А. Г.
          \end{minipage}
        \end{flushright}
        \vspace{\fill}
        \begin{center}
          Волгоград, \the\year
        \end{center}
    \end{titlepage}
    \setcounter{page}{2}
	\section*{Введение}
	В данной работе определяются характеристики отражательного клистрона К-97Р: его рабочая частота и добротность резонатора. Все измерения проводятся в "горячем" режиме.
	
	\section{Теоретическая часть}
	\subsection{Устройство отражательного клистрона и принцип его работы}
	Отражательный клистрон представляет собой резонансный генератор колебаний СВЧ малой мощности.
	Работа отражательного клистрона основана на кратковременном взаимодействии электрического поля резонатора с электронным потоком. 
	Электроны,  эмиттируемые  катодом,  ускоряются  в  пространстве  между катодом и резонатором, к которому приложено ускоряющее напряжение $U_0$.
	
	Возникающее между сетками резонатора CBЧ-напряжение $U(t)=U_m \sin \omega t$ производит модуляцию скорости электронов. В кинетической
	теории клистрона доказывается следующее приближённое выражение для скорости на выходе из модулирующего промежутка:
	\begin{equation}
	v = v_0 \left(1 + \frac{U_m}{2U_0} \beta \sin \left(\omega t_0 + \frac{\Phi}{2}\right) \right) = v_0 \left(1 + X \sin \left(\omega t_0 + \frac{\Phi}{2}\right) \right) ,
	\end{equation}
	где $v_0$, $t_0$~-- скорость электронов на влёте в резонатор и время влёта в резонатор соответственно, скорость определяется положительным потенциалом $U_0$ на резонаторе:
	\[
	\frac{mv_0^2}{2} = e U_0,
	\]
	$m$ и $e$ соответственно масса и абсолютное значение заряда электрона, $\beta$~-- параметр эффективной модуляции:
	\[
	\beta = \frac{\sin \cfrac{\Phi}{2}}{\cfrac{\Phi}{2}},
	\]
	$\Phi = d\omega/v_0$ угол пролёта электрона через высокочастотный зазор, $d$~-- расстояние между сетками резонатора.
	После вылета из резонатора электроны, двигаясь равнозамедленно в тормозящем поле отражателя $(U_0 + |U_r|)/l$, где $U_r$~-- потенциал отражателя, $l$~-- расстояние от резонатора до отражателя, уменьшают свою скорость до нулевого значения, затем начинают обратное движение и возвращаются в резонатор. В процессе этого движения к отражателю и обратно из-за различия скоростей электронов происходит образование сгустков. Движение в тормозящем поле происходит по параболам. Обозначив время вылета из резонатора $t_1$, получим уравнения траекторий в виде:
	\begin{equation}
	z = v (t - t_1) - \frac{e(U_0 + |U_r|)}{2ml} (t - t_1)^2.
	\end{equation}
	Найдём время возвращения в резонатор. Очевидно, что оно равно:
	\begin{equation}
	t_2 = t_1 +  v \frac{2ml}{e(U_0 + |U_r|)}.
	\end{equation}
	
	Чтобы  образовавшиеся  электронные  сгустки  отдавали  энергию  СВЧ-полю и поддерживали колебания в резонаторе, они должно возвращаться в резонатор в тормозящий полупериод. Для этого необходимо, как это видно из пространственно-временной диаграммы (рисунок \ref{cond}),  чтобы  сгустки  электронов  возвращались  в  резонатор  через целое число периодов без одной четверти, т.е. 
	$$ \omega (t_2 - t_1) = 2\pi \left(n + \frac{3}{4}\right)$$
	где n = 0, 1, 2, 3, 4, ... 
	
	\begin{figure}[h]
		\center
		\includegraphics[width = 0.6\textwidth]{images/12.png}
		\caption{Условие генерации}
		\label{cond}
	\end{figure}
	
	Отсюда следует формула для расчёта зон генерации:
	\begin{equation}
	n \approx \frac{\omega mlv_0}{\pi e(U_0 + |U_r|)} - \frac{3}{4} \approx \sqrt{\frac{2m}{e}}\frac{\omega l\sqrt{U_0}}{\pi (U_0 + |U_r|)} - \frac{3}{4}.
    \label{eq:n}
	\end{equation}
	
	Воспользовавшись представлениями об эквивалентной схеме клистрона \colorbox{yellow}{[]} или более точным методом, в котором решается уравнение возбуждения \colorbox{yellow}{[]}, можно получить следующие соотношения для частоты генерируемого сигнала $f$ и его мощности $P$ (рисунок \ref{figz}):
	\begin{gather}
	f = f_0 \left(1 - \frac{1}{2Q} \tan \left( \frac{2\pi(n + 3/4)}{U_0 + |U_r|} \delta U_r \right)\right), \label{eq:Q}\\
	P = I_0 U_0 \frac{\cos (2\pi (n + 3/4)\delta U_r/(U_0 + |U_r|)}{\pi (n + 3/4)} X J_1(X),
	\end{gather}
	где $f_0 = \omega/2\pi$~-- частота соответствующая центру зоны генерации, $Q$~-- добротность резонатора, $\delta U_r$~-- перестройка напряжения от центра зоны генерации, $I_0$~-- ток резонатора, $J_1(X)$~-- функция Бесселя первого порядка, $X = U_m\beta/2U_0$~-- параметр модуляции. Отсюда следует, что для определения добротности резонатора необходимо определить перестройку частоты в малом диапазоне напряжений отражателя в окрестности центра зоны генерации.
	
	\begin{figure}[h]
		\center
		\includegraphics[width = 0.8\textwidth]{images/zones.png}
		\caption{Условие генерации}
		\label{figz}
	\end{figure}
	
	\section{Методика проведения измерений}
	\subsection{Описание установки}
	В работе используются следующие приборы: клистрон, блок питания клистрона, анализатор спектра С4-27, ваттметр поглощаемой мощности, термисторная головка к ваттметру, а также соединительные элементы (2 волноводно-коаксиальных переходника, коаксиальный кабель,  коаксиальные аттенюаторы (используются при необходимости)). Последовательно собираются две схемы измерений.
	
	Первая схема измерений предназначена для определения напряжений отражателя, соответствующих зонам генерации. Она представлена на рисунке \ref{sch1}.
    \begin{figure}[h]
        \center
        \begin{tikzpicture}
            \begin{scope}
                % корпус
                \draw (-1, -1) -- (-1, 1);
                \draw (1, -1) -- (1, 1);
                \draw (1, 1) arc (0:180:1 and 0.7); 
                \draw (-1, -1) arc (180:360:1 and 0.7);
                % резонатор (выход)
                \draw (0.7, -0.25) arc (-150:150:0.5);
                \fill (0.7, -0.25) circle (2pt);
                \fill (0.7, 0.25) circle (2pt);
                \draw ({0.7 + 0.1 + 0.5*(1 + sqrt(3)/2)}, 0) circle (0.1);
                % выход
                \draw ({0.7 + 0.2 + 0.5*(1 + sqrt(3)/2)}, 0) -- (3, 0);
                \draw (2.4 , -0.3) rectangle (2.7, 0.3);
                % нагреватель
                \draw (-0.4, -2.5) -- (-0.4, -1.5);
                \draw (0.4, -2.5) -- (0.4, -1.5);
                \draw (0.4, -1.5) arc (0:180: 0.4 and 0.3);
                \draw (-0.4,-2.5) to[sV] (0.4, -2.5);
                % катод
                \draw (-0.6, -2.5) -- (-0.6, -1.5);
                \draw (-0.6, -1.5) arc (180:45: 0.6 and 0.5);
                % отражатель
                \draw (0, 2) -- (0, 1.2);
                \draw (0.4, 0.8) -- (0, 1.2) -- (-0.4, 0.8);
                % резонатор
                \draw (-0.5, -0.5) -- (-0.5, -0.3) -- (-0.1, -0.3);
                \draw (0.5, -0.5) -- (0.5, -0.3) -- (0.1, -0.3);
                \draw (-1.5, -0.4) -- (-0.5, -0.4);
                \draw (-1.5, -2.5) -- (-3, -2.5) -- (-3, -1.45) node[component]{$\mathrm{V_1}$} -- (-3, -0.4) -- (-1.5, -0.4);
                \node at (-2, -1.45) {$U_0$};
                \draw (-0.6, -2.5) -- (-1.5, -2.5) to [battery1] (-1.5, -0.4);
                
                \draw (0, 2) -- (-4, 2) to [battery1] (-4, -2.5) -- (-1.5, -2.5);
                \node at (-3.5, -.25) {$U_r$};
                \draw (-4, 2) -- (-5, 2) -- (-5, -.25) node[component]{$\mathrm{V_2}$} -- (-5, -2.5) -- (-1.5, -2.5);
            \end{scope}
            \draw (3,0) -- (3.5, 0);

            % переход на коаксиал
            \draw (3, -0.6) -- (3, 0.6);
            \draw (2.7, 0.6) -- (3.3, 0.6);
            \draw (2.7, 0) -- (4.5,0);
            \draw (3.6,0) circle (0.3cm);
            \draw (3.3,-0.3) -- (3.9,-0.3);

            % как обозначить анализатор?
            \draw (4.5, -0.5) rectangle (4.5 + 1, 0.5);
            \node at (5, 0) {$\alpha$};

            \draw (7, -0.6) -- (7, 0.6);
            \draw (6.7, 0.6) -- (7.3, 0.6);
            \draw (5.5, 0) -- (8.5,0);
            \draw (6.4,0) circle (0.3cm);
            \draw (6.1,-0.3) -- (6.7,-0.3);
            \draw (7.3 , -0.3) rectangle (7.6, 0.3);
            % ваттметр
            \draw (9, 0) circle (.5);
            \node at (9, 0) {$\mathrm{W}$};
        \end{tikzpicture}
        \caption{Функциональная схема установки для определения частоты клистрона.}
        \label{sch1}
    \end{figure}

	Вторая схема измерений предназначена для определения частот в пределах зон генерации и представлена на рисунке \ref{sch2}.

    \begin{figure}[h]
        \center
        \begin{tikzpicture}
            %клистрон
                       \begin{scope}
                % корпус
                \draw (-1, -1) -- (-1, 1);
                \draw (1, -1) -- (1, 1);
                \draw (1, 1) arc (0:180:1 and 0.7); 
                \draw (-1, -1) arc (180:360:1 and 0.7);
                % резонатор (выход)
                \draw (0.7, -0.25) arc (-150:150:0.5);
                \fill (0.7, -0.25) circle (2pt);
                \fill (0.7, 0.25) circle (2pt);
                \draw ({0.7 + 0.1 + 0.5*(1 + sqrt(3)/2)}, 0) circle (0.1);
                % выход
                \draw ({0.7 + 0.2 + 0.5*(1 + sqrt(3)/2)}, 0) -- (3, 0);
                \draw (2.4 , -0.3) rectangle (2.7, 0.3);
                % нагреватель
                \draw (-0.4, -2.5) -- (-0.4, -1.5);
                \draw (0.4, -2.5) -- (0.4, -1.5);
                \draw (0.4, -1.5) arc (0:180: 0.4 and 0.3);
                \draw (-0.4,-2.5) to[sV] (0.4, -2.5);
                % катод
                \draw (-0.6, -2.5) -- (-0.6, -1.5);
                \draw (-0.6, -1.5) arc (180:45: 0.6 and 0.5);
                % отражатель
                \draw (0, 2) -- (0, 1.2);
                \draw (0.4, 0.8) -- (0, 1.2) -- (-0.4, 0.8);
                % резонатор
                \draw (-0.5, -0.5) -- (-0.5, -0.3) -- (-0.1, -0.3);
                \draw (0.5, -0.5) -- (0.5, -0.3) -- (0.1, -0.3);
                \draw (-1.5, -0.4) -- (-0.5, -0.4);
                \draw (-1.5, -2.5) -- (-3, -2.5) -- (-3, -1.45) node[component]{$\mathrm{V_1}$} -- (-3, -0.4) -- (-1.5, -0.4);
                \node at (-2, -1.45) {$U_0$};
                \draw (-0.6, -2.5) -- (-1.5, -2.5) to [battery1] (-1.5, -0.4);
                
                \draw (0, 2) -- (-4, 2) to [battery1] (-4, -2.5) -- (-1.5, -2.5);
                \node at (-3.5, -.25) {$U_r$};
                \draw (-4, 2) -- (-5, 2) -- (-5, -.25) node[component]{$\mathrm{V_2}$} -- (-5, -2.5) -- (-1.5, -2.5);
            \end{scope}
            
            % переход на коаксиал
            \draw (3, -0.6) -- (3, 0.6);
            \draw (2.7, 0.6) -- (3.3, 0.6);
            \draw (2.7, 0) -- (4.5,0);
            \draw (3.6,0) circle (0.3cm);
            \draw (3.3,-0.3) -- (3.9,-0.3);

            % как обозначить анализатор?
            \draw (4.5, -0.5) rectangle (4.5 + 1, 0.5);
            \node at (5, 0) {$\alpha$};
            \draw (5.5, 0) -- (6.5,0);
            \draw (6.5, -0.5) rectangle (6.5 + 1, 0.5);
            \node at (7, 0) {$\mathrm{S}$};

        \end{tikzpicture}
        \caption{Функциональная схема установки для определения частоты клистрона.}
        \label{sch2}
    \end{figure}
	
	\subsection{Описание приборов}
	\subsubsection{Клистрон}
	
	Параметры клистрона по паспорту, который есть в лаборатории, представлены в таблице \ref{tab1}.
	
	\begin{table}[h]
		\center
        \caption{Параметры клистрона}
        \label{tab1}
		\begin{tabular}{|p{7cm}|c|c|c|}
			\hline
			\centering Параметры прибора
			& не менее
			& номинал
			& не более \\ \hline
			\centering Напряжение накала, В
			& $6{,}0$
			& $6{,}3$
			& $6{,}6$ \\ \hline
			\centering Напряжение резонатора, В
			& $345$
			& $350$
			& $355$ \\ \hline
			\centering Напряжение отражателя, В
			& $30$
			& $90..190$
			& $200$ \\ \hline
			\centering Ток накала, А
			& $0{,}6$
			& 
			& $1{,}55$ \\ \hline
			\centering Ток катода, мА
			& 
			&
			& $55$ \\ \hline
			\centering Время готовности, с
			&
			&
			& $90$ \\ \hline
		\end{tabular}
	\end{table}
	
	Цвета, номера и назначение проводов клистрона представлены в таблице \ref{tab2}
	\begin{table}[h]
		\center
        \caption{Цвета и назначение проводов клистрона}
        \label{tab2}
		\begin{tabular}{|c|c|c|}
			\hline
			Цвет 	& Номер	& Назначение \\ \hline
			белый 	& 	7	& нагреватель\\ \hline
			зелёный & 	2	& катод и нагреватель\\ \hline
			красный & 	6	& общий(резонатор)\\ \hline
			жёлтый  & 	1	& отражатель\\ \hline
		\end{tabular}
	\end{table}
	
	Нагрев катода осуществляется через белый и зелёный провода. Жёлтый провод идёт к отражателю, а красный провод соединяется с землёй и идёт к резонатору. Второй белый провод соединён с зелёным проводом и предназначен для подачи отрицательного напряжения $-350\,$В на катод. Можно не соединять белый провод с отрицательным напряжением, а с помощью дополнительного провода соединить зелёный провод с выходом $-350\,$В. Перед началом работы требуется проверить, что ни один из белых проводов не оторван от зелёного провода, для этого нужно убедиться в наличии тока в цепи нагревателя. 
	
	\subsubsection{Блок питания клистрона}
	
	Блок питания клистрона представлен на рисунке \ref{figbp}. Клеммы 1 отвечают за напряжение отражателя, клемма 2~-- земля(резонатор), клеммы 3 напряжение резонатора. Ручка 4 регулирует напряжение отражателя, ручка 7 регулирует напряжение катода. Клеммы 6 отвечают за нагрев катода. Тумблер 5 переключает показания вольтметра 10 с напряжения отражателя на напряжение катода. Тумблеры 8 включают-выключают напряжения отражателя, нагрева и катода. Переключатель 9 включает-выключает блок питания.
	
    \begin{figure}[h]
    \center
	\begin{tikzpicture}[scale=.08 ,
  circuit ee IEC]
    \tikzstyle{every node}=[font=\small]
  	% контур
	\draw[rounded corners,draw=black,thick](0,0)
    rectangle (100,100);
    % клеммы
    \draw[draw=black,thick](30,90) circle (4);
    \draw[draw=black,thick](30,90) circle (2);
    \draw[draw=black,thick](40,90) circle (4);
    \draw[draw=black,thick](40,90) circle (2);
    \draw[draw=black,thick](50,90) circle (4);
    \draw[draw=black,thick](50,90) circle (2);
    \draw[draw=black,thick](60,90) circle (4);
    \draw[draw=black,thick](60,90) circle (2);
    \draw[draw=black,thick](70,90) circle (4);
    \draw[draw=black,thick](70,90) circle (2);
    \draw[draw=black,thick](80,80) circle (3);
    \draw[draw=black,thick](80,80) circle (1.5);
    \draw[draw=black,thick](90,80) circle (3);
    \draw[draw=black,thick](90,80) circle (1.5);
    % тумблеры
    % переключение вольтметра
    \draw[black,thick](50, 78) circle (3);
    \draw[black,thick](50, 78) circle (2);
    \fill[fill=white,rounded corners,draw=black,thick](49,77)
    rectangle (56,79);
    % включение напряжений + индикаторы
    \draw[black,thick](80, 50) circle (3);
    \draw[black,thick](90, 50) circle (3);
    \draw[black,thick](90, 50) circle (2);
    \fill[fill=white,rounded corners,draw=black,thick](89,51)
    rectangle (91,46);
    \node[scale=1] at (90,43) {$6.3$};


    \draw[black,thick](80, 35) circle (3);
    \draw[black,thick](90, 35) circle (3);
    \draw[black,thick](90, 35) circle (2);
    \fill[fill=white,rounded corners,draw=black,thick](89,36)
    rectangle (91,31);
    \node[scale=1] at (90,28) {$-190$};


    \draw[black,thick](80, 20) circle (3);
    \draw[black,thick](90, 20) circle (3);
    \draw[black,thick](90, 20) circle (2);
    \fill[fill=white,rounded corners,draw=black,thick](89,21)
    rectangle (91,16);
    \node[scale=1] at (90,13) {$-350$};

    % основной тумблер
    \draw[black,thick](20, 10) rectangle (25,20);
    \draw[black](20, 15) -- (25,15);
    \node[scale=1] at (22.5,12.5) {$0$};
    \node[scale=1] at (22.5,17.5) {$1$};

    % потенциометр отражателя
    \draw[draw=black,thick](20,80) circle (5);

    % потенциометр резонатора
    \draw[thick](100,79)
    rectangle (106,81);


    % вольтметр
    \draw[rounded corners,draw=black,thick](30,10) rectangle (70,60);
    \draw[draw=black,thick](32,22) rectangle (68,58);
    \begin{scope}[xshift=50cm,yshift=16cm,scale=25]
        \foreach \i in {60,63,...,120} \draw (\i:1.15)--(\i:1.20);
        \foreach \i in {60,75,...,120} \draw[thick] (\i:1.15)--(\i:1.25);
        \node[scale=.7] at (60:1.05) {$400$};
        \node[scale=.7] at (75:1.35) {$300$};
        \node[scale=.7] at (90:1.35) {$200$};
        \node[scale=.7] at (105:1.35) {$100$};
        \node[scale=.7] at (120:1.05) {$0$};
        \draw[very thick] (95:0.25) -- (95:1);
        \draw (95:1) -- (95:1.20);
        \node[scale=1.5] at (90:0.7) {$\mathrm{V}$};
    \end{scope}

    % подписи
    % клеммы
    \node[scale=1.25] at (30,83) {$-$};
    \node[scale=1.25] at (40,83) {$+$};
    \node[scale=1.25] at (35,78) {$190$};

    \draw (50,84.5) to (50, 83) node[ground,rotate=-90,xshift=3.9ex,scale=6] {};

    \node[scale=1.25] at (60,83) {$-$};
    \node[scale=1.25] at (70,83) {$+$};
    \node[scale=1.25] at (65,78) {$350$};
    \node[scale=1] at (84,75) {$\sim 6.3$};

    % обозначения
    \draw (37, 93) -- (20,105) -- (15,105) node[above right] {$1$};
    \draw (30, 94) -- (30,98);

    \draw (47, 93) -- (30,105) -- (25,105) node[above right] {$2$};

    \draw (67, 93) -- (50,105) -- (45,105) node[above right] {$3$};
    \draw (60, 94) -- (60,98);

    \draw (16,82) -- (-5, 90) -- (-10, 90) node[above right] {$4$};
    
    \draw (47,77) -- (-5, 60) -- (-10, 60) node[above right] {$5$};

    \draw (82, 82) -- (105,105) -- (110,105) node[above left] {$6$};
    \draw (90, 83) -- (90,90);

    \draw (106, 80) -- (110, 90) -- (115,90) node[above left] {$7$};

    \draw (93, 50) -- (105, 40) -- (110,40) node[above left] {$8$};
    \draw (93, 35) -- (105, 40);
    \draw (93, 20) -- (105, 40);

    \draw (22.5, 10) -- (15, -10) -- (10,-10) node[above right] {$9$};

    \draw (50, 10) -- (70, -10) -- (75,-10) node[above left] {$10$};
	\end{tikzpicture}
	\caption{Блок питания клистрона}
	\label{figbp}
	\end{figure}
	
	\subsubsection{Измеритель мощности}

	В работе используется ваттметр проходной мощности М3-22А (представлен на рисунке \ref{figm}), подключаемый к волноводному выходу с помощью измерительного преобразователя М5-40 (термисторная головка). Измеритель мощности М3-22А предназначен для измерения среднего значения мощности непрерывных и импульсно-модулированных СВЧ-сигналов и коаксиальных и волноводных трактах. Индикаторный блок прибора М3-22А имеет автоматические установку нуля и выбор пределов измерения.
	
	\begin{figure}[h]
		\center
		\includegraphics[width = \textwidth]{images/m3-22a.png}
		\caption{Измеритель мощности М3-22А}
		\label{figm}
	\end{figure}
	
	Принцип действия прибора основан на автоматическом замещении поглощаемой мощности термистором СВЧ мощности, эквивалентной по тепловому воздействию, мощностью постоянного тока, а также на эквивалентности измерения сопротивления термистора при нагреве его рабочего тела постоянным током и током СВЧ.
	
	Основные технические данные М3-22А:
	\begin{enumerate}
		\item	Диапазон частот: 0,03~--~53,6~ГГц.
		\item	Пределы измерения мощности: 1 мкВт~--~10 мВт.
		\item	Волноводы: 35$\times$15, 28$\times$12, 23$\times$10, 17$\times$8, 11$\times$5,5, 7,5$\times$3,4, 5,2$\times$2,6 мм.
	\end{enumerate}
	
	Назначение ваттметра~-- определение зон генерации, по этой причине показания ваттметра роли не играют и в работе не снимаются. 
	
	\subsubsection{Анализатор спектра}
	
	В работе для определения частоты применяется анализатор спектра С4-27. Внешний вид анализатора спектра представлен на рисунке \ref{figa}.

    \begin{figure}[!h]
        \center
        \includegraphics[width = \textwidth]{images/s4-27.png}
        \caption{Анализатор спектра С4-27}
        \label{figa}
    \end{figure}

	Анализатор спектра С4-27 предназначен для исследования спектров периодически повторяющихся импульсов и непрерывных сигналов. Диапазон частот прибора от 0,01 до 39,6~ГГц с разбивкой на пять поддиапазонов: 0,01~--~1,9; 1,9~--~12; 12~--~17; 17~--~26; 26~--~39,6~ГГц. Прибор обеспечивает свои характеристики после времени прогрева не менее одного часа, но при широкой полосе обзора, когда не требуется высокая точность измерений достаточно 10 минут. На вход смесителей~-- непосредственно на ВХОД GHz, для волноводных~-- в крайнем положении ручки ОСЛАБЛЕНИЕ нельзя подавать сигнал мощностью более 1~мВт. В этом случае сигнал на прибор следует подавать только через аттенюатор прибора или внешний аттенюатор. Максимальная мощность, подаваемая на входные аттенюаторы приборы не должна превышать 0,2~Вт. Для клистрона с волноводным выходом $23\times10\,$мм, используется диапазон шкалы 1,9~--~12~ГГц частотами > 6,52~ГГц.  
	
	Ручкой НАСТРОЙКА находят диапазон, в котором есть отклики сигнала, выставляя на блоке СВЧ частоту в нужном диапазоне на некоторое деление, затем переходят в режим меток 10 МГц на блоке ПЧ, после чего отсчитывают количество меток от центральной метки до исследуемого сигнала на экране и вычитают или прибавляют полученное значение к значению по шкале блока СВЧ. Точность такого измерения составляет 10~МГц. Её можно повысить, если уменьшить полосу пропускания исследуемого сигнала и воспользоваться метками с шагом 0,1~ГГц и 1~ГГц.
	
	\subsubsection{Аттенюаторы}
	
	В работе используются 3 коаксиальных аттенюатора. Так как с самого начала уровень выходной мощности не известен, то требуется последовательно подключая аттенюаторы на 10 дБ, 6 дБ и 3 дБ, провести измерения уровня мощности до тех пор пока он не окажется в пределах 10 мВт, также аттенюаторы следует применить при измерении частоты по анализатору спектра, но здесь важно следить за тем, чтобы сигнал можно было отличить от шума. 

	\subsection{Проведение измерений}

	\subsubsection{Определение зон генерации, рабочей частоты и добротности}
	
	\begin{enumerate}
		\item Проверьте, что на рабочем столе присутствуют 3 соединительных провода, клистрон, блок питания, коаксиальный кабель, 2 волноводно-коаксиальных переходника, волновод (сечением больше либо равный сечению выходного фланца клистрона), термисторная головка для данного волновода, измеритель мощности, вольтметр (или мультиметр).
		\item Соберите установку по схеме измерений \ref{sch2}. Включите измеритель мощности и дайте ему прогреться. Подключите к нему термисторную головку, и соедините её с волноводом нужного сечения или сразу с волноводно-коаксиальным переходником. Переходник соедините с аттенюатором пока на 10 дБ. Аттенюатор соедините с коаксиальным кабелем, а коаксиальный кабель с клистроном.
		\item Включите блок питания и проверьте, что напряжения катода и отражателя соответствуют номинальным напряжениям таблицы \ref{tab1}. Выключите блок питания.
		\item Выставьте 0 на измерителе мощности с помощью кнопок грубой и точной настройки (первая и вторая кнопка соответственно, в режиме грубой настройки воспользуется винтом и добейтесь максимально близкого к 0 значения, затем воспользуйтесь кнопкой точной настройки), нажмите третью кнопку для того чтобы перейти в режим измерений.
		\item На блоке питания переключите тумблер вольтметра 5 в режим напряжений катода. Включите блок питания. Когда показания вольтметра начнут падать (подождите 90~c.), начнётся генерация клистрона. Переключите тумблер 5 в режим напряжений отражателя и проверьте наличие генерации во всём диапазоне от 90 до 190~В, если показания измерителя мощности слишком малы во всём диапазоне напряжений, измените аттенюатор на другой с меньшим затуханием. Если и это не дало результата, уберите аттенюатор. Обычно в этом случае в центрах зон генерации показания ваттметра достигают нескольких мВт.
		\item Занесите в таблицу \ref{zones1} границы зон генерации $U_{r, min}$ $U_{r, max}$, центры зон генерации $U_{r, \text{ц}}$ по вольтметру 10 или внешнему вольтметру и количество зон генерации в данном диапазоне напряжений отражателя. Запишете также напряжения на катоде в этих точках. 
		\begin{table}[h]
			\center
			\caption{Определение зон генерации}
			\label{zones1}
			\begin{tabular}{|c|c|c|c|c|}\hline
				\multirow{2}*{$U_0,~\text{В}$} & \multicolumn{3}{|c|}{$|U_r|,~\text{В}$} & \multirow{2}*{n} \\ \cline{2-4}
				& минимум & центр & максимум & \\
				\hline
				\multirow{2}*{} &  &  & &  \\ \hline
				&  &  &  & \\ \hline
			\end{tabular}
		\end{table}
		\item Используя формулу \eqref{eq:n} для напряжений в центрах зон генерации определите номера полученных зон.
		\item Отсоедините коаксиальный кабель с аттенюатором от переходника к термисторной головке и подсоедините его ко входу анализатора спектра 1,9~--~12~ГГц через аттенюатор 10 дБ. Ручкой настройка найдите сигнал. Меняя аттенюаторы и ручкой ТОК СМЕСИТЕЛЯ, добейтесь максимального уровня сигнала.
		\item Наблюдая спектр на экране анализатора, определите наименьшую частоту~-- это и будет основная частота клистрона.
		\item Изменяя напряжение на отражателе снимите зависимость частоты от напряжения отражателя. Занесите данные в таблицу \ref{freqs1}
		\begin{table}[h]
			\center
			\caption{Определение основной частоты и добротности}
			\label{freqs1}
			\begin{tabular}{|c|c|c|c|c|}\hline
				$U_0,~\text{В}$ & $|U_r|,~\text{В}$ & $f,~\text{ГГц}$ & $f_0,~\text{ГГц}$ & $Q$ \\ \hline
				\multirow{7}*{}
				&& &
				\multirow{7}*{}&\multirow{7}*{}\\ \cline{1-3}
				&& &&\\ \cline{1-3}
				&& &&\\ \cline{1-3}
				&& &&\\ \cline{1-3}
				&& &&\\ \hline
			\end{tabular}
		\end{table}
		\item Используя формулу \eqref{eq:Q} определите добротность клистрона.
	\end{enumerate}


    \section{Результаты}
    На этапе подготовки выяснилось, что имеющийся клистрон К-97Р со стандартным волноводным выходом не работает~-- на его выходе не обнаруживается СВЧ-сигнал, а через 5 минут после начала работы в цепи катода прекращает течь ток. Остальные клистроны К-97Р, имеющиеся в нашем распоряжении, работали нормально, однако, не обладали подходящим волноводным выходом. Попытки отсоединить волноводный выход от неисправного клистрона не увенчались успехом, поэтому измерения производились на единственном рабочем клистроне со стандартным выходным окном К-54А.

    В ходе выполнения работы были определены зоны генерации и для одной из зон была снята зависимость частоты генерируемого сигнала от напряжения отражателя. Эти данные представлены в таблицах~\ref{zones} и \ref{freqs} и на рисунке~\ref{figres}.

    Напряжение на резонаторе \( U_0 \) в процессе измерения при вращении соответствующей ручки на источнике питания почти не изменялось, поэтому во всех опытах оно одинаково. Для получения на отражателе напряжения, меньшего 80~В использовался резисторный делитель.

    \begin{table}[h]
        \center
        \caption{Определение зон генерации}
        \label{zones}
        \begin{tabular}{|c|c|c|c|c|}\hline
              \multirow{2}*{$U_0,~\text{В}$} & \multicolumn{3}{|c|}{$|U_r|,~\text{В}$} & \multirow{2}*{n} \\ \cline{2-4}
              & минимум & центр & максимум & \\
              \hline
              \multirow{3}*{180} & 61 & 79 & 93 & 6 \\
              & 105 & 128 & 146 & 5 \\
              & 182 & 202 & -- & 4 \\ \hline
        \end{tabular}
    \end{table}

    Для определения электронной перестройки была выбрана зона с $ n = 5 $, так как она полностью помещалась в диапазон напряжений источника питания.

    \begin{table}[h]
        \center
        \caption{Определение основной частоты и добротности}
        \label{freqs}
        \begin{tabular}{|c|c|c|c|c|}\hline
            $U_0,~\text{В}$ & $|U_r|,~\text{В}$ & $f,~\text{ГГц}$ & $f_0,~\text{ГГц}$ & $Q$ \\ \hline
            \multirow{7}*{180}
            &145& 8,27&
            \multirow{7}*{8,22$\pm$0,01}&\multirow{7}*{200$\pm$50}\\
            &140& 8,24&&\\
            &132& 8,23&&\\
            &127& 8,22&&\\
            &123& 8,21&&\\
            &117& 8,20&&\\
            &110& 8,19&&\\ \hline
        \end{tabular}
    \end{table}

    \begin{figure}[h]
        \center
        \begin{tikzpicture}
            \tikzstyle{every node}=[font=\small]
            \begin{axis}[ height=9cm, width=9cm, grid=major, xmin=60, xmax=200, ymin=8.18, ymax=8.28, xlabel={$|U_r|,~\text{В}$}, ylabel={$f,~\text{ГГц}$}]

            \addplot[name path=f1,white] coordinates {(61,8) (61,9)};
            \addplot[name path=f2,white] coordinates {(93,8) (93,9)};
            \addplot[name path=f3,white] coordinates {(105,8) (105,9)};
            \addplot[name path=f4,white] coordinates {(146,8) (146,9)};
            \addplot[name path=f5,white] coordinates {(182,8) (182,9)};
            \addplot[name path=f6,white] coordinates {(200,8) (200,9)};

            \addplot[thick,color=blue,fill=blue,fill opacity=0.1] fill between[of=f1 and f2];
            \addplot[thick,color=blue,fill=blue,fill opacity=0.1]fill between[of=f3 and f4];
            \addplot[thick,color=blue,fill=blue,fill opacity=0.1]fill between[of=f5 and f6];

            \addplot[color=black,smooth,very thick] coordinates {
                    (145, 8.27)
                    (140, 8.24)
                    (132, 8.23)
                    (127, 8.22)
                    (123, 8.21)
                    (117, 8.20)
                    (110, 8.19)
                    };
            \addplot[only marks,mark=point,error bars/.cd,y dir=both,y explicit,x dir=both,x explicit] coordinates {
                    (145, 8.27) +- (1, 0.01)
                    (140, 8.24) +- (1, 0.01)
                    (132, 8.23) +- (1, 0.01)
                    (127, 8.22) +- (1, 0.01)
                    (123, 8.21) +- (1, 0.01)
                    (117, 8.20) +- (1, 0.01)
                    (110, 8.19) +- (1, 0.01)
                    };
            \end{axis}
        \end{tikzpicture}
        \caption{Зависимость частоты от напряжения на отражателе для одной из зон генерации; зоны генерации отмечены цветом}
        \label{figres}
    \end{figure}
\end{document}