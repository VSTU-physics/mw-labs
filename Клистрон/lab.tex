\documentclass[a4paper]{article}
\usepackage[russian]{babel}
\usepackage[utf8]{inputenc}
\usepackage[margin=1cm,left=2cm]{geometry} % сделать по ГОСТУ
\usepackage{amsmath}
\usepackage{tikz}
\usetikzlibrary{circuits.ee.IEC}

\renewcommand{\tan}{\mathrm{tg\,}}
\begin{document}
	% не забыть вставить титульник
	\section*{Введение}
	В данной работе определяются характеристики отражательного клистрона К-97Р: его рабочая частота и добротность резонатора. Все измерения проводятся в "горячем" режиме.
	
	\section{Теоретическая часть}
	\subsection{Устройство отражательного клистрона и принцип его работы}
	Отражательный клистрон представляет собой резонансный генератор колебаний СВЧ малой мощности.
	Работа отражательного клистрона основана на кратковременном взаимодействии электрического поля резонатора с электронным потоком. 
	Электроны,  эмиттируемые  катодом,  ускоряются  в  пространстве  между катодом и резонатором, к которому приложено ускоряющее напряжение $U_0$.
	
	Возникающее между сетками резонатора CBЧ-напряжение $U(t)=U_m \sin \omega t$ производит модуляцию скорости электронов. В кинетической
	теории клистрона доказывается следующее приближённое выражение для скорости на выходе из модулирующего промежутка:
	\begin{equation}
	v = v_0 \left(1 + \frac{U_m}{2U_0} \beta \sin \left(\omega t_0 + \frac{\Phi}{2}\right) \right) = v_0 \left(1 + X \sin \left(\omega t_0 + \frac{\Phi}{2}\right) \right) ,
	\end{equation}
	где $v_0$, $t_0$ -- скорость электронов на влёте в резонатор и время влёта в резонатор соответственно, скорость определяется положительным потенциалом $U_0$ на резонаторе:
	\[
	\frac{mv_0^2}{2} = e U_0,
	\]
	$m$ и $e$ соответственно масса и абсолютное значение заряда электрона, $\beta$ -- параметр эффективной модуляции:
	\[
	\beta = \frac{\sin \cfrac{\Phi}{2}}{\cfrac{\Phi}{2}},
	\]
	$\Phi = d\omega/v_0$ угол пролёта электрона через высокочастотный зазор, $d$ -- расстояние между сетками резонатора.
	После вылета из резонатора электроны, двигаясь равнозамедленно в тормозящем поле отражателя $(U_0 + |U_r|)/l$, где $U_r$ -- потенциал отражателя, $l$ -- расстояние от резонатора до отражателя, уменьшают свою скорость до нулевого значения, затем начинают обратное движение и возвращаются в резонатор. В процессе этого движения к отражателю и обратно из-за различия скоростей электронов происходит образование сгустков. Движение в тормозящем поле происходит по параболам. Обозначив время вылета из резонатора $t_1$, получим уравнения траекторий в виде:
	\begin{equation}
	z = v (t - t_1) - \frac{e(U_0 + |U_r|)}{2ml} (t - t_1)^2.
	\end{equation}
	Найдём время возвращения в резонатор. Очевидно, что оно равно:
	\begin{equation}
	t_2 = t_1 +  v \frac{2ml}{e(U_0 + |U_r|)}.
	\end{equation}
	
	Чтобы  образовавшиеся  электронные  сгустки  отдавали  энергию  СВЧ-полю и поддерживали колебания в резонаторе, они должно возвращаться в резонатор в тормозящий полупериод. Для этого необходимо, как это видно из пространственно-временной диаграммы \colorbox{yellow}{рис.},  чтобы  сгустки  электронов  возвращались  в  резонатор  через целое число периодов без одной четверти, т.е. 
	$$ \omega (t_2 - t_1) = 2\pi \left(n + \frac{3}{4}\right)$$
	где n = 0, 1, 2, 3, 4, ... 
	
	Отсюда следует формула для расчёта зон генерации:
	\begin{equation}
	n \approx \frac{mlv_0}{\pi e(U_0 + |U_r|)} - \frac{3}{4} \approx \sqrt{\frac{2m}{e}}\frac{l\sqrt{U_0}}{\pi (U_0 + |U_r|)} - \frac{3}{4}.
	\end{equation}
	
	Воспользовавшись представлениями об эквивалентной схеме клистрона \colorbox{yellow}{[]} или более точным методом, в котором решается уравнение возбуждения \colorbox{yellow}{[]}, можно получить следующие соотношения для частоты генерируемого сигнала $f$ и его мощности $P$:
	\begin{gather}
	f = f_0 \left(1 - \frac{1}{2Q} \tan \left( \frac{2\pi(n + 3/4)}{U_0 + |U_r|} \delta U_r \right)\right), \\
	P = I_0 U_0 \frac{\cos (2\pi (n + 3/4)\delta U_r/(U_0 + |U_r|)}{\pi (n + 3/4)} X J_1(X),
	\end{gather}
	где $f_0$ -- частота соответствующая центру зоны генерации, $Q$ -- добротность резонатора, $\delta U_r$ -- перестройка напряжения от центра зоны генерации, $I_0$ -- ток резонатора, $J_1(X)$ -- функция Бесселя первого порядка, $X = U_m\beta/2U_0$ -- параметр модуляции.
	
	\section{Методика проведения измерений}
	\subsection{Описание установки}
	Так как мы поссорились с нужным клистроном с волноводным переходником и он больше не хочет работать дольше пяти минут, то мы изменили ему с К-54. Все дальнейшие измерения проводились на последнем.

	Для определения характеристик клистрона необходимо узнать при каких напряжениях на резонаторе и отражателе начинается генерация. Для этого необходимо подключить клистрон к измерителю мощности.

	По-моему 2 схемы, учитывая многочисленность компонентов, рисовать бессмысленно и можно просто остановиться на функциональной.

	Далее в пределах одной зоны можно найти основную частоту и засчёт электронной перестройки найти добротность.
	
	\subsection{Описание приборов}
	\subsubsection{Блок питания клистрона}
    \begin{figure}[h]
    \center
	\begin{tikzpicture}[
  scale=.08 ,
  circuit ee IEC]
  	% контур
	\draw[rounded corners,draw=black,thick](0,0)
    rectangle (100,100);
    % клеммы
    \draw[draw=black,thick](30,90) circle (4);
    \draw[draw=black,thick](30,90) circle (2);
    \draw[draw=black,thick](40,90) circle (4);
    \draw[draw=black,thick](40,90) circle (2);
    \draw[draw=black,thick](50,90) circle (4);
    \draw[draw=black,thick](50,90) circle (2);
    \draw[draw=black,thick](60,90) circle (4);
    \draw[draw=black,thick](60,90) circle (2);
    \draw[draw=black,thick](70,90) circle (4);
    \draw[draw=black,thick](70,90) circle (2);
    \draw[draw=black,thick](80,80) circle (3);
    \draw[draw=black,thick](80,80) circle (1.5);
    \draw[draw=black,thick](90,80) circle (3);
    \draw[draw=black,thick](90,80) circle (1.5);
    % тумблеры
    % переключение вольтметра
    \draw[black,thick](50, 78) circle (3);
    \draw[black,thick](50, 78) circle (2);
    \fill[fill=white,rounded corners,draw=black,thick](49,77)
    rectangle (56,79);
    % включение напряжений + индикаторы
    \draw[black,thick](80, 50) circle (3);
    \draw[black,thick](90, 50) circle (3);
    \draw[black,thick](90, 50) circle (2);
    \fill[fill=white,rounded corners,draw=black,thick](89,51)
    rectangle (91,46);
    \node[scale=1] at (90,43) {$6.3$};


    \draw[black,thick](80, 35) circle (3);
    \draw[black,thick](90, 35) circle (3);
    \draw[black,thick](90, 35) circle (2);
    \fill[fill=white,rounded corners,draw=black,thick](89,36)
    rectangle (91,31);
    \node[scale=1] at (90,28) {$-190$};


    \draw[black,thick](80, 20) circle (3);
    \draw[black,thick](90, 20) circle (3);
    \draw[black,thick](90, 20) circle (2);
    \fill[fill=white,rounded corners,draw=black,thick](89,21)
    rectangle (91,16);
    \node[scale=1] at (90,13) {$-350$};

    % основной тумблер
    \draw[black,thick](20, 10) rectangle (25,20);
    \draw[black](20, 15) -- (25,15);
    \node[scale=1] at (22.5,12.5) {$0$};
    \node[scale=1] at (22.5,17.5) {$1$};

    % потенциометр отражателя
    \draw[draw=black,thick](20,80) circle (5);

    % потенциометр резонатора
    \draw[thick](100,79)
    rectangle (106,81);


    % вольтметр
    \draw[rounded corners,draw=black,thick](30,10) rectangle (70,60);
    \draw[draw=black,thick](32,22) rectangle (68,58);
    \begin{scope}[xshift=50cm,yshift=16cm,scale=25]
        \foreach \i in {60,63,...,120} \draw (\i:1.15)--(\i:1.20);
        \foreach \i in {60,75,...,120} \draw[thick] (\i:1.15)--(\i:1.25);
        \node[scale=.7] at (60:1.05) {$400$};
        \node[scale=.7] at (75:1.35) {$300$};
        \node[scale=.7] at (90:1.35) {$200$};
        \node[scale=.7] at (105:1.35) {$100$};
        \node[scale=.7] at (120:1.05) {$0$};
        \draw[very thick] (95:0.25) -- (95:1);
        \draw (95:1) -- (95:1.20);
        \node[scale=1.5] at (90:0.7) {$\mathrm{V}$};
    \end{scope}

    % подписи
    % клеммы
    \node[scale=1.25] at (30,83) {$-$};
    \node[scale=1.25] at (40,83) {$+$};
    \node[scale=1.25] at (35,78) {$190$};

    \draw (50,84.5) to (50, 83) node[ground,rotate=-90,xshift=3.9ex,scale=6] {};

    \node[scale=1.25] at (60,83) {$-$};
    \node[scale=1.25] at (70,83) {$+$};
    \node[scale=1.25] at (65,78) {$350$};
    \node[scale=1] at (84,75) {$\sim 6.3$};

    % обозначения
    \draw (37, 93) -- (20,105) -- (15,105) node[above right] {$1$};
    \draw (30, 94) -- (30,98);

    \draw (47, 93) -- (30,105) -- (25,105) node[above right] {$2$};

    \draw (67, 93) -- (50,105) -- (45,105) node[above right] {$3$};
    \draw (60, 94) -- (60,98);

    \draw (16,82) -- (-5, 90) -- (-10, 90) node[above right] {$4$};
    
    \draw (47,77) -- (-5, 60) -- (-10, 60) node[above right] {$5$};

    \draw (82, 82) -- (105,105) -- (110,105) node[above left] {$6$};
    \draw (90, 83) -- (90,90);

    \draw (106, 80) -- (110, 90) -- (115,90) node[above left] {$7$};

    \draw (93, 50) -- (105, 40) -- (110,40) node[above left] {$8$};
    \draw (93, 35) -- (105, 40);
    \draw (93, 20) -- (105, 40);

    \draw (22.5, 10) -- (15, -10) -- (10,-10) node[above right] {$9$};

    \draw (50, 10) -- (70, -10) -- (75,-10) node[above left] {$10$};
	\end{tikzpicture}
    \end{figure}
	\subsubsection{Измеритель мощности}
	\subsubsection{Анализатор спектра}

	\subsection{Проведение измерений}
	\subsubsection{Определение зон генерации}
    \begin{figure}[h]
        \center
        \begin{tikzpicture}
    %клистрон
    % корпус
    \draw (-1, -1) -- (-1, 1);
    \draw (1, -1) -- (1, 1);
    \draw (1, 1) arc (0:180:1 and 0.7); 
    \draw (-1, -1) arc (180:360:1 and 0.7);
    % резонатор (выход)
    \draw (0.7, -0.4) arc ({-90-atan(0.3/0.4)}:{90+atan(0.3/0.4)}:{sqrt(0.4^2+0.3^2)});
    \fill (0.7, -0.4) circle (2pt);
    \fill (0.7, 0.4) circle (2pt);
    \draw ({sqrt(0.4^2+0.3^2) + 1 + 0.1}, 0) circle (0.1);
    % выход
    \draw ({sqrt(0.4^2+0.3^2) + 1 + 0.2}, 0) -- ({sqrt(0.4^2+0.3^2) + 1 + 1}, 0);
    \draw ({0.5 + 1 + 0.4 , -0.4}) rectangle ({0.5 + 1 + 0.8}, 0.4);
    % нагреватель
    \draw (-0.4, -2) -- (-0.4, -1.5);
    \draw (0.4, -2) -- (0.4, -1.5);
    \draw (0.4, -1.5) arc (0:180: 0.4 and 0.3);
    % катод
    \draw (-0.6, -2) -- (-0.6, -1.5);
    \draw (-0.6, -1.5) arc (180:45: 0.6 and 0.5);
    % отражатель
    \draw (0, 2) -- (0, 1.2);
    \draw (0.4, 0.8) -- (0, 1.2) -- (-0.4, 0.8);
    \draw (-1.5, -0.4) -- (-0.5, -0.4);
    % резонатор
    \draw (-0.5, -0.5) -- (-0.5, -0.3) -- (-0.1, -0.3);
    \draw (0.5, -0.5) -- (0.5, -0.3) -- (0.1, -0.3);
    
    %соединение волноводов
    \draw (2.5, 0) -- (4, 0);
    \draw (3, -0.4) -- (3, .4);
    \draw (3.2, -0.4) -- (3.2, .4);
    % ваттметр
    \draw (4.4, 0) circle (.4);
    \node at (4.4, 0) {$\mathrm{W}$};

        \end{tikzpicture}
    \end{figure}
	\subsubsection{Определение рабочей частоты и добротности}
        \begin{figure}[h]
        \center
        \begin{tikzpicture}
    %клистрон
    % корпус
    \draw (-1, -1) -- (-1, 1);
    \draw (1, -1) -- (1, 1);
    \draw (1, 1) arc (0:180:1 and 0.7); 
    \draw (-1, -1) arc (180:360:1 and 0.7);
    % резонатор (выход)
    \draw (0.7, -0.4) arc ({-90-atan(0.3/0.4)}:{90+atan(0.3/0.4)}:{sqrt(0.4^2+0.3^2)});
    \fill (0.7, -0.4) circle (2pt);
    \fill (0.7, 0.4) circle (2pt);
    \draw ({sqrt(0.4^2+0.3^2) + 1 + 0.1}, 0) circle (0.1);
    % выход
    \draw ({sqrt(0.4^2+0.3^2) + 1 + 0.2}, 0) -- ({sqrt(0.4^2+0.3^2) + 1 + 1}, 0);
    \draw ({0.5 + 1 + 0.4 , -0.4}) rectangle ({0.5 + 1 + 0.8}, 0.4);
    % нагреватель
    \draw (-0.4, -2) -- (-0.4, -1.5);
    \draw (0.4, -2) -- (0.4, -1.5);
    \draw (0.4, -1.5) arc (0:180: 0.4 and 0.3);
    % катод
    \draw (-0.6, -2) -- (-0.6, -1.5);
    \draw (-0.6, -1.5) arc (180:45: 0.6 and 0.5);
    % отражатель
    \draw (0, 2) -- (0, 1.2);
    \draw (0.4, 0.8) -- (0, 1.2) -- (-0.4, 0.8);
    \draw (-1.5, -0.4) -- (-0.5, -0.4);
    % резонатор
    \draw (-0.5, -0.5) -- (-0.5, -0.3) -- (-0.1, -0.3);
    \draw (0.5, -0.5) -- (0.5, -0.3) -- (0.1, -0.3);
    
    %соединение волноводов
    \draw (2.5, 0) -- (4, 0);
    \draw (3, -0.4) -- (3, .4);
    \draw (3.2, -0.4) -- (3.2, .4);
    % как обозначить анализатор?
    \draw (4.4, 0) circle (.4);
    \node at (4.4, 0) {$\mathrm{S}$};

        \end{tikzpicture}
    \end{figure}
\end{document}