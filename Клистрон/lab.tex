\documentclass[a4paper]{article}
\usepackage[russian]{babel}
\usepackage[utf8]{inputenc}
\usepackage[margin=1cm,left=2cm]{geometry} % сделать по ГОСТУ
\begin{document}
	% не забыть вставить титульник
	\section*{Введение}
	В этой работе определяются характеристики отражательного клистрона К-97Р. Определяются его рабочая частота и добротность его резонатора.
	
	\section{Теоретическая часть}
	\subsection{Устройство отражательного клистрона}
	Катод, резонатор, отражатель и мыкающиеся электроны
	\subsection{Принцип работы отражательного клистрона}
	Модуляция, группировка, возвращение в нужную фазу. Кинетическая теория клистрона, условие генерации.
	
	\section{Методика проведения измерений}
	\subsection{Описание установки}
	Так как мы поссорились с нужным клистроном с волноводным переходником и он больше не хочет работать дольше пяти минут, то мы изменили ему с К-54. Все дальнейшие измерения проводились на последнем.

	Для определения характеристик клистрона необходимо узнать при каких напряжениях на резонаторе и отражателе начинается генерация. Для этого необходимо подключить клистрон к измерителю мощности.

	По-моему 2 схемы, учитывая многочисленность компонентов, рисовать бессмысленно и можно просто остановиться на функциональной.

	Далее в пределах одной зоны можно найти основную частоту и засчёт электронной перестройки найти добротность.
	
	\subsection{Описание приборов}
	\subsubsection{Блок питания клистрона}
	\subsubsection{Измеритель мощности}
	\subsubsection{Анализатор спектра}

	\subsection{Проведение измерений}
	\subsubsection{Определение зон генерации}

	\subsubsection{Определение рабочей частоты и добротности}
\end{document}